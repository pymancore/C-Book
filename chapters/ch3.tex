\chapter{استاندارهای C}
استانداردهای زبانی به مجموعه‌ای از قواعد زبانی گفته می‌شوند که به برنامه‌نویس‌ها کمک می‌کند برنامه‌ای که می‌نویسند را در هر جایی بتوانند اجرا کنند. برای درک بهتر این موضوع می‌توانید به قواعد نگارشی دستور زبان فارسی فکر کنید. دستور زبان فارسی بیان می‌کند که برای هر ضمیر، فعل مناسب آن ضمیر بکار رود. بعنوان مثال طبق دستور زبان فارسی جمله \textbf{من می‌روم} یک جمله درست است اما جمله \textbf{من می‌روند} یک جمله نادرست است. فایده این قواعد زبانی آن است که همگان را مجاب به رعایت آن می‌نماید و در نتیجه با رعایت این قوانین از طرف همه افراد، می‌توانیم به راحتی حرف بزنیم و دیگران نیز به راحتی متوجه گفته‌های ما شوند.\\
فرض کنید چنین قواعدی وجود نداشتند، در این صورت نمی‌توانستیم به منظور گفته‌های همدیگر پی ببریم. در زبان‌های برنامه نویسی نیز بهمین شکل است. فرض کنید کسی بجای آکولاد پرانتز بگذارد یا بجای دو نقطه از علامت سوال استفاده کند. در آن صورت نمی‌توانیم برنامه‌های جهان شمولی تولید کنیم زیرا کسی منظور برنامه‌های ما را در نمی‌یابد و ما نیز منظور آنان را در نمی‌یابیم. 
\section{استاندارد K\&R}
\section{استاندارد ANSI/ISO}
\section{استاندارد C99}
\section{استاندارد C11}
\section{استاندارد C18}
