\chapter{تاریخچه زبان C}
کامپیوترهای قدیمی به گونه‌ای بودند که در یک بازه زمانی مشخص فقط به یک کاربر اجازه کار میدادند. در سال ۱۹۶۴ دانشگاه MIT با مشارکت دو شرکت بزرگ جنرال الکتریک\LTRfootnote{General Electric} و AT\&T\LTRfootnote{American Telephone \& Telegraph} پروژه‌ای بنام Multics\LTRfootnote{Multiplexed Information and Computing Service } را شروع کردند. مولتیکس سیستم عاملی بود که امکان اشتراک زمانی\LTRfootnote{Time sharing} را فراهم میکرد. در اینگونه سیستم عاملها، کاربران می‌توانند همزمان از منابع یک سیستم استفاده کنند. این سیستم عامل هرچند که در نوع خود بی‌نظیر بود اما معایبی نیز داشت. یکی از معایبی که باعث شد AT\&T از این پروژه خارج شود، قیمت این سیستم عامل بود. با وجود مزایای فراوان مولتیکس، اما قیمت این سیستم عامل فراتر از امکاناتی بود که ارائه میداد. دنیس ریچی\LTRfootnote{Dennis Ritchie} بهمراه چند تن از همکارانش بعنوان کارمندان AT\&T از افراد درگیر در پروژه مولتیکس بودند.\\پس از خروج AT\&T از پروژه مولتیکس،‌ دنیس ریچی به دو تن از همکاران خود به نام‌های کِن تامسون\LTRfootnote{Ken Thompson} و برایان کرنیگان\LTRfootnote{Brian Kernighan} در آزمایشگاه‌های بِل\LTRfootnote{Bell Labs} متعلق به شرکت AT\&T ملحق شد. پروژه‌ای که این افراد به آن مشغول بودند، ساخت یک فایل سیستم \LTRfootnote{file system} برای مینی‌کامپیوتر PDP-7\LTRfootnote{Programmed Data Processor} متعلق به شرکت DEC\LTRfootnote{Digital Equipment Corporation} بود. کن تامسون توانسته بود با زبان اسمبلی \LTRfootnote{assembly}فایل سیتم PDP-7 را بسازد. او به یاری دو همکار دیگرش تلاش کردند تا این فایل سیتم را بهبود دهند. تلاش برای بهبود این فایل سیستم منجر به خلق سیستم عامل یونیکس\LTRfootnote{UNIX} شد. نام Unics\LTRfootnote{Uniplexed Information and Computing Service} در واقع صرفا بازی با کلمات در تقابل با مولتیکس بود که توسط کرنیگان بر روی سیتم‌عامل جدید گذاشته شد. وی این نام را به Unix تغییر داد و ایده وی این بود که هیچکس با دیدن نام Unix به ریشه اصلی آن پی نخواهد برد. این سیستم عامل تا نگارش سوم به زبان اسمبلی توسعه یافت. فهم زبان اسمبلی برای انسان سخت است اما در آن زمان این باور وجود داشت که سیستم عامل نمی‌تواند به زبانی جز اسمبلی نوشته شود. به منظور بکارگیری و تسهیل در استفاده از یونیکس همچنین از زبان‌های فرترن\LTRfootnote{FORTRAN} و B نیز استفاده شده بود.

سیستم‌عامل یونیکس بکلی به زبان اسمبلی نوشته شده بود. کاربران مجبور بودند برای دادن فرامین به کامپیوتر، چندین صفحه برنامه را به زبان اسمبلی بنویسند. اما این یک پروسه طاقت‌فرسا بود. به همین دلیل در سیستم‌عامل یونیکس از مترجم‌های فرترن و B برای فرمان دادن به کامپیوتر استفاده می‌شد. زبان B توسط کن تامسون و دنیس ریچی بر پایه زبان BCPL\LTRfootnote{Basic Combined Programming Language} که توسط مارتین ریچاردز\LTRfootnote{Martin Richards} ابداع شده بود، ساخته شده بود. برخلاف زبان اسمبلی، کار کردن با زبان B راحتتر بود و برای نوشتن یک برنامه، نیازی نبود خطهای زیادی کد نوشته شود. اما باز هم زبان B دارای کاستی‌هایی بود که نمی‌توانست نیاز طراحان یونیکس را مرتفع کند. از مهمترین ایرادهای وارده به زبان B این بود که نوع داده‌ها\LTRfootnote{Data Types} را نمی‌شناخت یا اینکه در B چیزی بنام ساختمان داده‌ها\LTRfootnote{Data Structure} تعریف نشده بود. برای رفع این کاستی‌ها، ریچی و کرنیگان تصمیم گرفتند تا یک زبان جدید را از ریشه طراحی کنند. در خلال سالهای ۱۹۷۱ تا ۱۹۷۳ زبان Cتوسعه یافت. علت این نامگذاری این بود که زبان C اکثر خصوصیات خود را از B به ودیعه گرفته بود و بر پایه B ساخته شده بود. زبان C در اصل با این هدف ساخته شد که بتوانند سیستم عامل یونیکس را با آن پیاده کنند. این تلاش نتیجه داد و در سال ۱۹۷۳ تمام کدهای اسمبلی کرنل یونیکس را برای کامپیوتر PDP-11 با C نوشتند و یونیکس نسخه ۴ اولین یونیکسی بود که تماما با C نوشته شده بود.