\chapter{مفاهیم و تعاریف}
\section{تعریف برنامه‌نویسی}
برنامه‌نویسی به مجموعه‌ای از شیوه‌ها گفته می‌شود که از طریق آن دستورالعمل‌هایی را به کامپیوتر ارائه می‌دهیم تا کامپیوتر بر مبنای آنچه که تعریف کرده‌ایم برای ما کارهایی را انجام دهد. 
\subsection{برنامه‌نویسی و زبان برنامه‌نویسی}
بسیاری از افراد \textbf{برنامه‌نویسی } را با \textbf{زبان برنامه‌نویسی} اشتباه می‌گیرند در صورتی که اینگونه نیست. همچنان که قبلا توضیح دادیم، برنامه‌نویسی به مجموعه‌ای از شیوه‌ها بمنظور فرمان دادن به کامپیوتر گفته می‌شود، حال آنکه زبان برنامه‌نویسی، همچنان که از ذات کلمه برمی‌آید، زبانی برای انتقال دستورات ما به کامیپوتر است. در این‌جا با ذکر چند مثال ملموس این تفاوت را درمیابیم:\\
فرض کنید می‌خواهید یک ساندویچ درست کنید و از یک شخص دیگر می‌خواهید دستور تهیه آن ساندویچ را به شما بدهد. اگر شخص مورد نظر شما فارس زبان باشد، دستور تهیه را به زبان فارسی و اگر آلمانی باشد، دستور تهیه را به زبان آلمانی به شما می‌دهد. هر دو دستور منجر به تهیه ساندویچ یکسانی خواهد شد اما به زبان‌های متفاوت. در این مثال دستور آشپزی همان برنامه‌نویسی و زبان دستور، همان زبان برنامه‌نویسی است.
به طریق دیگر می‌توانیم برنامه‌نویسی را به مسیر و  زبان برنامه‌نویسی را به وسیله نقلیه تشبیه کنیم. یک مسیر را می‌توان با انواع مختلفی از وسایل نقلیه طی کرد، مانند هواپیما، قطار، کشتی، دوچرخه، موتور و یا حتی پیاده. مسئله‌ای که در اینجا حائز اهمیت است این است که هر وسیله نقلیه‌ای مناسب نوع خاصی از مسیر است. بعنوان مثال پیمودن مسافت بین دو کشور با پای پیاده چندان معقول نیست. در اینجا به نکته مهمی می‌رسیم و آن این است که هر گونه جنگی بین برتری زبان‌ها، جنگی بی نتیجه است زیرا همانطور که گفتیم هر زبان بسان یک وسیله نقلیه، مناسب نوع خاصی از مسیر است.
\section{تاریخچه زبان C}
کامپیوترهای قدیمی به گونه‌ای بودند که در یک بازه زمانی مشخص فقط به یک کاربر اجازه کار میدادند. در سال ۱۹۶۴ دانشگاه MIT با مشارکت دو شرکت بزرگ جنرال الکتریک\LTRfootnote{General Electric} و AT\&T\LTRfootnote{American Telephone \& Telegraph} پروژه‌ای بنام Multics\LTRfootnote{Multiplexed Information and Computing Service } را شروع کردند. مولتیکس سیستم‌عاملی بود که امکان اشتراک زمانی\LTRfootnote{Time sharing} را فراهم میکرد. در اینگونه سیستم‌عامل‌ها، کاربران می‌توانند همزمان از منابع یک سیستم استفاده کنند. این سیستم‌عامل هرچند که در نوع خود بی‌نظیر بود اما معایبی نیز داشت. یکی از معایبی که باعث شد AT\&T از این پروژه کناره‌گیری کند، قیمت این سیستم عامل بود. با وجود مزایای فراوان مولتیکس، اما قیمت این سیستم عامل فراتر از امکاناتی بود که ارائه میداد. دنیس ریچی\LTRfootnote{Dennis Ritchie} بهمراه چند تن از همکارانش بعنوان کارمندان AT\&T از افراد درگیر در پروژه مولتیکس بودند.\\پس از خروج AT\&T از پروژه مولتیکس،‌ دنیس ریچی به دو تن از همکاران خود به نام‌های کِن تامسون\LTRfootnote{Ken Thompson} و برایان کرنیگان\LTRfootnote{Brian Kernighan} در آزمایشگاه‌های بِل\LTRfootnote{Bell Labs} متعلق به شرکت AT\&T ملحق شد. پروژه‌ای که این افراد به آن مشغول بودند، ساخت یک فایل سیستم \LTRfootnote{file system} برای مینی‌کامپیوتر PDP-7\LTRfootnote{Programmed Data Processor} متعلق به شرکت DEC\LTRfootnote{Digital Equipment Corporation} بود. کن تامسون توانسته بود با زبان اسمبلی\LTRfootnote{assembly} فایل سیستم PDP-7 را بسازد. او به یاری دو همکار دیگرش تلاش کردند تا این فایل سیتم را بهبود دهند. تلاش برای بهبود این فایل سیستم منجر به خلق سیستم عامل یونیکس\LTRfootnote{UNIX} شد. نام Unics\LTRfootnote{Uniplexed Information and Computing Service} در واقع صرفا بازی با کلمات در تقابل با مولتیکس بود که توسط کرنیگان بر روی سیتم‌عامل جدید گذاشته شد. وی این نام را به Unix تغییر داد و ایده وی این بود که کسی با دیدن نام Unix به ریشه اصلی آن پی نخواهد برد. این سیستم عامل تا نگارش سوم به زبان اسمبلی توسعه یافت. فهم زبان اسمبلی برای انسان سخت است اما در آن زمان این باور وجود داشت که سیستم عامل نمی‌تواند به زبانی جز اسمبلی نوشته شود. به منظور بکارگیری و تسهیل در استفاده از یونیکس همچنین از زبان‌های فرترن\LTRfootnote{FORTRAN} و B نیز استفاده شده بود.

سیستم‌عامل یونیکس بکلی به زبان اسمبلی نوشته شده بود. کاربران مجبور بودند برای دادن فرامین به کامپیوتر، چندین صفحه برنامه را به زبان اسمبلی بنویسند. اما این یک پروسه طاقت‌فرسا بود. به همین دلیل در سیستم‌عامل یونیکس از مترجم‌های فرترن و B برای فرمان دادن به کامپیوتر استفاده می‌شد. زبان B توسط کن تامسون و دنیس ریچی بر پایه زبان BCPL\LTRfootnote{Basic Combined Programming Language} که توسط مارتین ریچاردز\LTRfootnote{Martin Richards} ابداع شده بود، ساخته شده بود. برخلاف زبان اسمبلی، کار کردن با زبان B راحتتر بود و برای نوشتن یک برنامه، نیازی نبود خطهای زیادی کد نوشته شود. اما باز هم زبان B دارای کاستی‌هایی بود که نمی‌توانست نیاز طراحان یونیکس را مرتفع کند. از مهمترین ایرادهای وارده به زبان B این بود که نوع داده‌ها\LTRfootnote{Data Types} را نمی‌شناخت یا اینکه در B چیزی بنام ساختمان داده‌ها\LTRfootnote{Data Structure} تعریف نشده بود. برای رفع این کاستی‌ها، ریچی و کرنیگان تصمیم گرفتند تا یک زبان جدید را از ریشه طراحی کنند. در خلال سالهای ۱۹۷۱ تا ۱۹۷۳ زبان C توسعه یافت. علت این نامگذاری این بود که زبان C اکثر خصوصیات خود را از B به ودیعه گرفته بود و بر پایه B ساخته شده بود. زبان C در اصل با این هدف ساخته شد که بتوانند سیستم عامل یونیکس را با آن بر روی ماشین‌های مختلفی پیاده کنند. از این رو دیگر نیازی نبود تا یونیکس را برای هر نوع سخت‌افزاری بصورت ویژه بازنویسی کنند، بلکه این کار را کامپایلر زبان C انجام می‌داد. به عبارت دیگر، C یک زبان مستقل از ماشین بود. این تلاش نتیجه داد و در سال ۱۹۷۳ تمام کدهای اسمبلی کرنل یونیکس را برای کامپیوتر PDP-11 با C نوشتند و یونیکس نسخه ۴ اولین یونیکسی بود که تماما با C نوشته شده بود.
\section{استاندارهای C}
استانداردهای زبانی به مجموعه‌ای از قواعد زبانی گفته می‌شوند که به برنامه‌نویس‌ها کمک می‌کند برنامه‌ای که می‌نویسند را در هر جایی بتوانند اجرا کنند. برای درک بهتر این موضوع می‌توانید به قواعد نگارشی دستور زبان فارسی فکر کنید. دستور زبان فارسی بیان می‌کند که برای هر ضمیر، فعل مناسب آن ضمیر بکار رود. بعنوان مثال طبق دستور زبان فارسی جمله \textbf{من می‌روم} یک جمله درست است اما جمله \textbf{من می‌روند} یک جمله نادرست است. فایده این قواعد زبانی آن است که همگان را مجاب به رعایت آن می‌نماید و در نتیجه با رعایت این قوانین از طرف همه افراد، می‌توانیم به راحتی حرف بزنیم و دیگران نیز به راحتی متوجه گفته‌های ما شوند.\\
فرض کنید چنین قواعدی وجود نداشتند، در این صورت نمی‌توانستیم به منظور گفته‌های همدیگر پی ببریم. در زبان‌های برنامه نویسی نیز بهمین شکل است. فرض کنید کسی بجای آکولاد پرانتز بگذارد یا بجای دو نقطه از علامت سوال استفاده کند. در آن صورت نمی‌توانیم برنامه‌های جهان شمولی تولید کنیم زیرا کسی منظور برنامه‌های ما را در نمی‌یابد و ما نیز منظور آنان را در نمی‌یابیم. اما استاندارهای C را چه کسانی تدوین می‌کنند؟ این استاندارها را هم‌اکنون دو سازمان به نام‌های کمیسیون الکتروتکنیکی بین‌المللی\LTRfootnote{International Electrotechnical Commission} (IEC) و سازمان بین‌المللی استانداردسازی\LTRfootnote{International Organization for Standardization} (ISO) طراحی و تدوین می‌کنند. سازندگان کامپایلرها استانداردها را از منابع رسمی دریافت می‌کنند و کامپایلر را مطابق دستورالعمل‌های آن می‌سازند. در نتیجه اگر شما به کامپایلر نیاز داشته باشید، باید از استانداردی پیروی کنید که کامپایلر مطابق آن ساخته شده است. در غیر این صورت کامپایلر توان فهم کدهای شما را ندارد.
\subsection{استاندارد K\&R}
در ابتدا استانداردی برای زبان C وجود نداشت. در سال ۱۹۷۸ که کرنیگان و ریچی کتاب «زبان برنامه‌نویسی C » را منتشر کردند، شیوه برنامه‌نویسی آنان بعنوان مرجعی برای شیوه برنامه‌نویسی به زبان C پذیرفته شد. همچنین در انتهای این کتاب ضمیمه‌ای تحت عنوان «راهنمای پیاده‌سازی C » وجود داشت که به استاندارد K\&R معروف شد. این دو حرف سرواژه‌های نام‌های کرنیگان و ریچی بود. استاندارد K\&R در حقیقت استاندارد نیست بلکه فقط یک شیوه است که تا مدتها از آن پیروی می‌شد.
\subsection{استاندارد ANSI/ISO}
\subsection{استاندارد C99}
\subsection{استاندارد C11}
\subsection{استاندارد C18}
\section{ویژگی‌های زبان C}
\subsection{طراحی}
\subsubsection{ساختار بالا به پایین}
\subsubsection{کتابخانه‌های استاندارد}
\subsubsection{برنامه‌نویسی ماژولار}
\subsection{بهینگی}
\subsection{قابلیت حمل}
\subsection{قدرت و انعطاف‌پذیری}
\subsection{آزادی}

