\part{مفاهیم و تعاریف}
\chapter{تعریف برنامه‌نویسی}
برنامه‌نویسی به مجموعه‌ای از شیوه‌ها گفته می‌شود که از  آن طریق دستورالعمل‌هایی را به کامپیوتر ارائه می‌دهیم تا کامپیوتر بر مبنای آنچه که تعریف کرده‌ایم برای ما کارهایی را انجام دهد. 
\section{برنامه‌نویسی و زبان برنامه‌نویسی}
بسیاری از افراد \textbf{برنامه‌نویسی } را با \textbf{زبان برنامه‌نویسی} اشتباه می‌گیرند در صورتی که اینگونه نیست. همچنان که قبلا توضیح دادیم، برنامه‌نویسی به مجموعه‌ای از شیوه‌ها بمنظور فرمان دادن به کامپیوتر گفته می‌شود، حال آنکه زبان برنامه‌نویسی، همچنان که ذات کلمه برمی‌آید، زبانی برای انتقال دستورات ما به کامیپوتر است. در این‌جا به ذکر چند مثال ملموس این تفاوت را درمیابیم:\\
فرض کنید می‌خواهید یک ساندویچ درست کنید و از یک شخص دیگر می‌خواهید دستور تهیه آن ساندویچ را به شما بدهد. اگر شخص مورد نظر شما فارس زبان باشد، دستور تهیه را به زبان فارسی و اگر آلمانی باشد، دستور تهیه را به زبان آلمانی به شما می‌دهد. هر دو دستور منجر به تهیه ساندویچ یکسانی خواهد شد اما به زبان متفاوت. در این مثال دستور همان برنامه‌نویسی و زبان دستور، همان زبان برنامه‌نویسی است.
به طریق دیگر می‌توانیم برنامه‌نویسی را به مسیر و  زبان برنامه‌نویسی را به وسیله نقلیه تشبیه کنیم. یک مسیر را می‌توان با انواع مختلفی از وسایل نقلیه طی کرد، مانند هواپیما، قطار، کشتی، دوچرخه، موتور و یا حتی پیاده. مسئله‌ای که در اینجا حائز اهمیت است این است که هر وسیله نقلیه‌ای مناسب نوع خاصی از مسیر است. بعنوان مثال پیمودن مسافت بین دو کشور با پای پیاده چندان معقول نیست. در اینجا به نکته مهمی می‌رسیم و آن این است که هر گونه جنگی بین برتری زبان‌ها، جنگی بی نتیجه است زیرا همانطور که گفتیم هر زبان بسان یک وسیله نقلیه، مناسب نوع خاصی از مسیر است.