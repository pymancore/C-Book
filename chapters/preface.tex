\chapter*{پیشگفتار}
امروزه زبانهای برنامه‌نویسی بی‌شماری وجود دارند که هرکدام بنابر ویژگی‌های خاص خود در جاهای خاص و برای اهداف بخصوصی استفاده میشوند. اما به جرأت می‌توان گفت که در میان این زبان‌ها هیچ زبانی به اندازه زبان C موفق نبوده است. در دنیایی که در حال تکاپوست و هر روز دستاوردهای جدید شیوه‌ها و رویه‌های قدیمی را از گردونه حیات به در میرانند، C اما برای حدود پنج دهه همچنان یکه‌تاز و در زمره برترین و پرکاربردترین زبان‌هاست. در دنیای رقابتی امروز که بر سر همه چیز اعم از کیفیت و قدرت و سرعت رقابت است، ماندگاری C بی‌دلیل نیست و برنامه‌نویسان بی‌جهت به اسفاده از یک زبان برنامه‌نویسی ادامه نمی‌دهند.

در دنیا منابع غنی بساری برای یادگیری این زبان وجود دارد. اما آنچه که سبب ایجاد این کتاب شده است در وهله اول نبود کتابی جامع و تفصیلی به زبان فارسی و در وهله دوم نبود کتابی با هدف تفهیم عمیق مفاهیم به زبان ساده حتی به زبان انگلیسی برای نوآموزان است. از این رو با تشویق و حمایت دوستان بر آن شدم تا دانسته‌های خود و دوستان را در یک کتاب به زبان فارسی گردآوردی و تالیف کنیم. هدف از تالیف این کتاب پر کردن فاصله خالی بین آموزش‌های مقدماتی و پیشرفته است. چه بسا بسیاری از نوآموزان به دلیل گنگ بودن کتابهای مقدماتی و نیافتن پاسخ‌های فلسفی(چرایی)شان در همان هفته‌ها و روزهای اول قید این زبان را می‌زنند. بنظر نگارنده واقعیت این است که فهم زبان C به فهم و درک دیگر زبان‌ها کمک بسزایی می‌نماید، چرا که با فهم و درک اصول برنامه‌نویسی در زبان C به درک ساختار و شیوه برنامه‌نویسی در دیگر زبانها نیز کمک می‌کند. از این رو در تالیف این کتاب علاوه بر توضیح مسائل متدوال برنامه‌نویسی، به توضیح چرایی هرکدام از پدیده‌های تازه نیز می‌پردازیم. 

این کتاب به رایگان در اختیار شما دانش‌پژوهان و علاقه‌مندان قرار میگیرد. امید است شما نیز با انتشار دسترسی همگان را به این کتاب فراهم نمائید.
